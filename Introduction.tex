\section{Introduction}
\label{sec:intro}
%
The increasing need for fast and flexible commissioning ($i.\,e.,$ order picking and collection of
unstructured goods from storage compartments in warehouses) in logistic scenarios has created
substantial interest for autonomous robotic solutions. One of the main arguments for automating this
task is that the dull and strenuous nature of commissioning could cause mental and physical illness
in human workers. As a result, the determination within the logistics sector to invest in this area
is high and substantial efforts are made for the humanization of workstations~\cite{Eche08}.

There exist partial solutions for the automated commissioning problem in controlled
environments. The system described in~\cite{Wurm08} coordinates a fleet of Autonomous Ground
Vehicles (AGVs) which transport shelves filled with goods to a human worker who picks the
corresponding objects to complete the order. Key obstacles for a fully automatized solution
applicable in general warehouse settings are the safe autonomous vehicle navigation in industrial
environments co-populated by humans, as well as the autonomous grasping/manipulation of unstructured
goods at reasonable cycle times.
%
\begin{figure}[t!]
\begin{center}
\includegraphics[width =0.85\linewidth]{figs/apple_demonstrator}
%\vspace{-0.25cm}
\caption{\textit{The APPLE platform:} A KUKA LBR iiwa arm (3) is mounted on a retrofitted Linde
  CitiTruck AGV (6). For localization, a Velodyne Lidar\protect\footnotemark (4) is used, human
  worker detection is carried out with the RefleX camera system (5). The depicted grasping device
  (2) is a further developed and smaller version of the Velvet Fingers gripper described
  in~\cite{Tinc12}. Each of the gripper’s two fingers has a planar manipulator structure with two
  rotary joints and active surfaces which are implemented by conveyor belts on the inside of the two
  phalanges. The mechanical structure of each finger is underactuated and comprises one actuated
  Degree of Freedom (DoF) for opening and closing and two DoF for the belt movements. If, during
  grasping, the proximal phalanges are blocked by an object, the gripper’s distal phalanges continue
  to ``wrap around'' and envelope it in a firm grasp. Object and pallet detection is done with a
  Structure IO device (1) which is mounted on the gripper's palm.}
\label{fig:robot}
\vspace{-0.65cm}
\end{center}
\end{figure}
\footnotetext{\url{http://www.velodynelidar.com/}}

In this work, we present the Autonomous Picking \& Palletizing (APPLE) research platform (see
Fig.~\ref{fig:robot}) which we developed to address the following important sub-task chain which
occurs during commissioning in prototypical warehouses: autonomous picking of goods from a storage
location, subsequent placement on a standard EUR half-pallet and transport of the filled pallet to a
target location. Furthermore, this process has to be carried out in a manner which is safe for
humans operating in the same environment. The platform's mobile base consists of a non-holonomic
Linde CitiTruck forklift AGV\footnote{\url{ http://www.citi-truck.com}} which is able to detect and
pick up pallets in designated loading zones \hl{@Henrik: is that a reasonable one-sentence
  description...?}. A KUKA LBR
iiwa\footnote{\url{http://www.kuka-labs.com/en/service_robotics/lightweight_robotics/}}.
leight-weight arm which is fitted with an under-actuated gripper with conveyor belts on the inside
of each finger is used for robust grasping and object manipulation. While this setup is not intended
as a close-to-market solution for palletizing, the force/torque sensing capabilities as well as the
compliant and human-safe control possibilities allow us to investigate various grasping/manipulation
strategies.

  In this paper, we outline our solution to the safe navigation of the APPLE platform in
an industrial scenario co-populated by human workers wearing reflective vests. Furthermore, we
introduce a novel grasp planning- and representation scheme which is utilized for reactive,
on-the-fly manipulator motion generation. This allows to exploit manipulator redundancies and offers
several advantages to the commonly used sense-plan-act architectures. As a final contribution, we
discuss our compliant grasp execution strategy which uses the active surfaces on the employed
gripper to increase grasp robustness.

The remainder of this article is organized as follows: below we outline the AGV navigation and
motion planning scheme before presenting our solution for people detection in
Section~\ref{subsec:AGV_related_work}. In Section~\ref{sec:manip} our approach to robust, online
grasp motion generation and grasp execution is discussed. We show early results and an evaluation of
the APPLE system by means on a simplified commissioning task in Section~\ref{sec:eval} before
drawing conclusions in Section~\ref{sec:discussion}.
