\section{Introduction}
\label{sec:intro}
%
The increasing need for fast and flexible commissioning ($i.\,e.,$ order picking and collection of
unstructured goods from storage compartments in warehouses) in logistic scenarios has created
substantial interest for autonomous robotic solutions. One of the main arguments for automating this
task is that the dull and strenuous nature of commissioning could cause mental and physical illness
in human workers. As a result, the determination within the logistics sector to invest in this area
is high and substantial efforts are made for the humanization of workstations~\cite{Eche08}.

There exist partial solutions for the automated commissioning problem in controlled
environments. The system described in~\cite{Wurm08} coordinates a fleet of Autonomous Ground
Vehicles (AGVs) which transport shelves filled with goods to a human worker who picks the
corresponding objects to complete the order. Key obstacles for a fully automatized solution
applicable in general warehouse settings are safe autonomous vehicle navigation in industrial
environments co-populated by humans, as well as the autonomous grasping/manipulation of unstructured
goods at satisfactory cycle times.
%
\begin{figure}[t!]
\begin{center}
\includegraphics[width =0.8\linewidth]{figs/apple_demonstrator}
%\vspace{-0.25cm}
\caption{\textit{The APPLE platform:} A KUKA LBR iiwa arm (3) is mounted on a retrofitted Linde
  CitiTruck AGV (7). For localization, a Velodyne HDL-32 Lidar (4) is used, human co-workers are
  detected with our RefleX camera system (6) presented in~\cite{Mosb14}. In order to detect pallets,
  we employ an Asus Xtion Pro Live structured light camera (5). The depicted grasping device (2) is
  a further developed and smaller version of the underactuated Velvet Fingers gripper described
  in~\cite{Tinc12}. Each of the gripper’s two fingers has a planar manipulator structure with two
  rotary joints and active surfaces which are implemented by conveyor belts on the inside of the two
  phalanges. These belts are used to assist in robust grasp acquisition as described in
  Section~\ref{subsec:grasp_execution}. Object and pallet detection is done with a Structure IO
  device (1) which is mounted on the gripper's palm.}
\label{fig:robot}
\vspace{-0.65cm}
\end{center}
\end{figure}

In this work, we present the Autonomous Picking \& Palletizing (APPLE) research platform (see
Fig.~\ref{fig:robot}) which we developed to address the following important sub-task chain which
occurs during commissioning in prototypical warehouses: autonomous picking of goods from a storage
location, subsequent placement on a standard EUR pallet and transport of the filled pallet to a
target location. Furthermore, this process has to be carried out in a manner which is safe for
humans operating in the same environment. The platform's mobile base consists of a retrofitted
nonholonomic Linde CitiTruck forklift AGV\footnote{\url{ http://www.citi-truck.com}} which is able
to detect and pick up pallets in designated loading zones. A KUKA LBR
iiwa\footnote{\url{http://www.kuka-labs.com/en/service_robotics/lightweight_robotics/}} lightweight
arm which is fitted with an under-actuated gripper with conveyor belts on the inside of each finger
is used for robust grasping and object manipulation. The APPLE platform is intended for research and
not as a close-to-market solution for palletizing. Therefore, the set of sensors used for navigation
and the manipulator were selected to allow exploring the possibilities for fully autonomous
commissioning, and not in an attempt on an economic solution.

In this paper, we outline our solution to the safe navigation of the APPLE platform in an industrial
scenario co-populated by human workforce wearing reflective safety clothing. Furthermore, we
introduce a novel grasp representation/planning scheme which is utilized for reactive, on-the-fly
manipulator motion generation. It allows to exploit manipulator redundancies and offers several
advantages to the commonly used sense-plan-act architectures. As a final contribution, we discuss
our compliant grasp execution strategy, which uses the active surfaces on the employed gripper to
increase grasp robustness.

The remainder of this article is organized as follows: in Section~\ref{sec:agv} we outline the AGV
navigation and motion planning scheme, as well as our solution for human detection in
Section~\ref{subsec:people_det}. Section~\ref{sec:manip} is dedicated to the developed grasping and
manipulation system with focus on object detection in Section~\ref{subsec:obj_det}, grasp
representation/planning in Section~\ref{subsec:grasp_planning} and reactive manipulator motion
generation in Section~\ref{subsec:manip_motion}. We then show early results and an evaluation of the
APPLE system in a simplified commissioning scenario in Section~\ref{sec:eval} before drawing
conclusions in Section~\ref{sec:discussion}.
