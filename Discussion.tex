\section{Discussion and Outlook}
\label{sec:discussion}
%
In this paper, we presented a research platform for autonomous picking and palletizing (APPLE) for
fully autonomous commissioning tasks in intralogistics settings. The APPLE robot comprises a
nonholonomic mobile base with the ability to autonomously detect and pick standard EUR half-pallets
from designated loading areas. Also incorporated is a camera system for detection and avoidance of
human workers wearing reflective vests. The manipulation system for loading/unloading unstructured
goods from pallets operates on a novel redundant grasp representation as intervals in task space
which allows to incorporate empirical knowledge. We leverage the obtained redundancy by generating
reactive manipulator motions on the fly using a prioritized control approach which allows to
formulate the target object picking as a stack of hierarchical tasks~\cite{Kano11}. We provide an
early experimental evaluation of the APPLE system by means of a simplified commissioning task (see
the video attachment to this article).

The presented work is limited to the picking of objects with cylindrical shapes, future work will
aim at extending our grasp interval representation to other primitive shape types. Also, we work on
a computational method to parametrize the interval for a current scene which, right now, is done
empirically. 


%\cite{Tass12}\cite{Kuma13}(optimal control for motion generation)