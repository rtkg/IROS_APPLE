\section{Autonomous Forklift}
\label{sec:agv}
%
%\hl{Introduce the CitiTruck + RefleX camera system}

The mobile platform is built upon a manual forklift ``CiTi'' from
Linde Material Handling. The forklift is originally equipped with a
motorized forks and drive wheel. The forklift has been retrofitted
with a steering mechanism and a commercial AGV control system which is
used to interface the original drive mechanism as well as the steering
servo.

To assure safe operation the vehicle is equipped with a standard safety
laser scanner (SICK S300) and an industrial prototype system (RefleX) for
detecting workforce using reflective safety garments. To show the workers the intention of the vehicle, the intend path to be driven with the required occupied is projected onto the floor. \hl{Should we mention the projector?}

%
\subsection{Challenges}
\label{subsec:AGV_challenges}

\hl{Remove this section, it will get really lengthy to be able to put in aspects of navigation, mapping, localization, safety here...}

The industry standard for autonomous navigation of forklifts is to use predefined
trajectories where the trajectories are either
manually defined or learned through teaching-by-demonstration from a human
operator~\cite{HellstromRingdahl.VAS06,MarshallEtAl.JFR08}.
Although conceptually simple, fixed trajectories limits the pallet handling to occur only at predefined fixed poses as well as simple strategies for handling unforeseen obstacles.

The fundamental difficulties for motion planning of forklifts lies in the the non-holonomic constraints, the large sweep area it needs to occupy (due to its very non symmetrical footprint) while operating in limited work space...


\subsection{Navigation}
\label{subsec:navigation}
%

The navigation modules ensures that the forklift is capable of moving save and autonomously through the work space environment to arbitrary load and unload poses with high accuracy. According to the AGV system provider Kollmorgen, the required end pose accuracy
for picking up pallets is $0.03$~m in position and $1$~degree (0.017 radians) in
orientation. The main component consist of trajectory generation, tacking controller and a localization system. 
 
The trajectory generation on-line is done in two steps, where at first a kinematically feasible path with discretized start and goal poses are generated using a lattice planner~\cite{Ciri14}. This path is post-processed using a path smoother~\cite{Andr15} which assure smooth collision-free continuous trajectory. The tracking of the trajectory is one using a model predictive tracking controller. The complete navigation system has been implemented, extensively tested and successfully
integrated on the APPLE demonstrator, a detailed description can be found in~\cite{Andr15}.

The localization utilize a Velodyne-32 3D laser scanner which is used to construct a 3D map (using the 3D-NDT-OM map representation) of the static parts of the environment~\cite{Stoy13}. The map and odometry information is used to localize the vehicle in the presence of dynamic entities using a dual timescale approach~\cite{Vale14}. 



\hl{Pallet detection?}

%
\subsection{People Detection}
\label{subsec:people_det}
%
As the envisioned mobile manipulation system will operate in environments shared with human workers,
people detection and human safety are important issues. In APPLE we address the problem by using the
RefleX system we recently developed~\cite{Mosb14}. RefleX is a camera-based on-board safety system
for industrial vehicles and machinery for detection of human workers wearing reflective vests worn
as per safety regulations. The system was designed with industrial safety standards in mind and is
currently being tested as an industrial prototype.
%
